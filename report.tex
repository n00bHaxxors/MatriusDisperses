\documentclass[11pt,a4paper,twoside]{report}
  \usepackage{a4wide}
  \usepackage{epsfig}
  \usepackage{amsmath}
  \usepackage{tabu}
  \usepackage{amsfonts}
  \usepackage{latexsym}
  \usepackage[utf8]{inputenc}
  \usepackage{listings}
  \usepackage{color}
  \usepackage{titlesec}    
  \usepackage{enumitem}
  \usepackage[catalan]{babel}
  \usepackage{newunicodechar}
  \usepackage{graphicx}
  \usepackage{subcaption}
  \usepackage{float}
  \newunicodechar{Ŀ}{\L.}
  \newunicodechar{ŀ}{\l.}
  
  
  % \titleformat{\chapter}
  %   {\normalfont\LARGE\bfseries}{\thechapter}{1em}{}
  % \titlespacing*{\chapter}{0pt}{3.5ex plus 1ex minus .2ex}{2.3ex plus .2ex}
  
  \definecolor{dkgreen}{rgb}{0,0.6,0}
  \definecolor{gray}{rgb}{0.5,0.5,0.5}
  \definecolor{mauve}{rgb}{0.58,0,0.82}
  
  \lstset{frame=tb,
    language=Matlab,
    aboveskip=3mm,
    belowskip=3mm,
    showstringspaces=false,
    columns=flexible,
    basicstyle={\small\ttfamily},
    numbers=none,
    numberstyle=\tiny\color{gray},
    keywordstyle=\color{blue},
    commentstyle=\color{dkgreen},
    stringstyle=\color{mauve},
    breaklines=true,
    breakatwhitespace=true,
    tabsize=3,
    extendedchars=true,
    literate={á}{{\'a}}1 {à}{{\`a}}1 {é}{{\'e}}1 {è}{{\`e}}1 {í}{{\'i}}1 {ó}{{\'o}}1 {ò}{{\`o}}1 {ú}{{\'u}}1
  }
  
  
  \usepackage{hyperref}
  \hypersetup{
      colorlinks=false, %set true if you want colored links
      linktoc=all,     %set to all if you want both sections and subsections linked
      linkcolor=blue,  %choose some color if you want links to stand out
  }
  
  
  \setlength{\footskip}{50pt}
  \setlength{\parindent}{0cm} \setlength{\oddsidemargin}{-0.5cm} \setlength{\evensidemargin}{-0.5cm}
  \setlength{\textwidth}{17cm} \setlength{\textheight}{23cm} \setlength{\topmargin}{-1.5cm} \addtolength{\parskip}{2ex}
  \setlength{\headsep}{1.5cm}
  
  \renewcommand{\contentsname}{Continguts}
  %\renewcommand{\chaptername}{Pr\`actica}
  \setcounter{chapter}{0}
  \begin{document}
  
  \title{Treball 1: Matrius disperses}
  \author{Ismael El Habri, Marc Cané, Lluís Trilla}
  \date{16 d'octubre de 2018}
  \maketitle
  
  \tableofcontents
  
  
  \chapter{Què són les matrius disperses?}
  
  Quan parlem de matrius disperses ens referim a matrius de gran tamany en la qual la majoria d'elements son zero. Direm que 
  una matriu és disperrsa, quan hi hagi benefici en aplicar els mètodes propis d'aquestes. 
  
  Per identificar si una matriu és dispersa, podem usar el seguent:
    
  \qquad Una matriu $n \times n$ serà dispersa si el número de coeficients no nuls es $n^{\gamma+1}$, on $\gamma < 1$.

  En funció del poblema, decidim el valor del paràmetre $\gamma$. Aquí hi ha els valors típics de $\gamma$:
  \begin{itemize}
    \item $\gamma=0.2$ per problemes d'anàlisi de sistemes eléctics degeneració i de transpot d'enegía.
    \item $\gamma=0.5$ per matrius en bandes associades a problemes d'anàlisi d'estructues.
  \end {itemize}



  Podem trobar dos tipus de matrius disperses:
  \begin{itemize}
    \item \textbf{Matrius estructurades:} matrius en les quals els elements diferents de zero formen un patró regular.
    \item \textbf{Matrius no estructurades:} els elements diferents de zero es distribueixen de forma irregular.
  \end{itemize}

  \chapter{Formes d'emmagatzemar matrius disperses}
  
  \section{Per Coordenades}
  
    És la primera aproximació que podriem pensar i és bastant intutiva. Per cada element no nul guardem una tupla amb el valor i les seves coordenades: $(a_{i j}, i, j)$. 
    
    \qquad \textbf{Exemple}
	\[    
    \begin{pmatrix}
    	1	&	0	& 0	&	2	\\
    	0	&	1	&	0	&	0	\\
    	0	&	0	&	0	&	0	\\
    	3	&	0	&	-2	&	0	\\
    \end{pmatrix}   
    \qquad = \qquad
	    \begin{tabu}{l|c}
	    	$índex$	&	$tupla$ (a_{i j}, i, j)	\\
	    	\hline
	    	0	&	(1, 0, 0) \\
	    	1	&	(2, 0, 3) \\
	    	2	& (1, 1, 1) \\
	    	3	& (1, 3, 0) \\
	    	4	& (-2, 3,2)	\\ 
	    \end{tabu}  
    \]
	
	\qquad Per emmagatzemar això podem usar tres vectors de la mateixa mida ($n_z$, el nombre d'elements diferents de zero): Un amb els valors, un amb les files i un amb les columnes:
	\begin{center}
		\begin{tabular}{l|c c c c c}
			Vector & \multicolumn{5}{c}{Coeficients}\\
			\hline
			valors			&	1	&	2	&	1 &	1	&	-2	\\
			files				&	0	&	0	&	1	&	3	&	3	\\
			columnes	&	0	&	3	&	1	&	0	&	2	\\ 	
		\end{tabular}	
	\end{center}
	
    
    A la realitat però, aquest mètode d'emmagatzemar les dades és poc eficient quan hem de fer operacions amb les matrius.
  
  %% es borrara
  Hi ha vàries formes de guardar aquestes matrius com ara:
  \begin{itemize}
    \item \textbf{Per coordenades.} És la primera aproximació que podriem pensar i és bastant intutiva. Per cada element no nul guardar el seu valor i la seva posició en la matriu (columna i fila).
    A la realitat però, aquest mètode d'emmagatzemar les dades és poc eficient quan hem de fer operacions amb les matrius.
    \item \textbf{Per files o columnes.} % tb es coneix com a Compressed sparse Rows (CSR) lequivalent pero amb columnes, es diu CSC
    Aquest mètode és similar a l'anterior però ara en comptes de guardar la fila i la columna de cada valor guardarem només una de les dos però afegirem una llista d'index per indicar el primer element de cada fila/columna. 
    Si decidim guardar per files per exemple guardarem el valor i la columna de cada element no nul i una llista amb l'index del primer element de cada fila.
    En cas de que tinguem una fila buida guardarem el mateix index que teniem a la fila anterior.
    Aquest és el mètode més habitual i el que implementarem.
    \item \textbf{Per perfil.} %També conegut com a esquema diagonal
     Aquest mètode és una manera eficient de guardar un tipus concret de matrius, les matrius banda.
    Aprofitant la propietat d'aquestes matrius aquest mètode guarda els valors de la banda de cada fila, els index del primer element no nul de cada fila i els index del primer valor de cada banda.
  \end{itemize}  
  %http://www.jldelafuenteoconnor.es/Clase_dispersa_2018.pdf

  %1.  Qu`e  ́es una matriu dispersa.  Definici ́o o definicions.
  %2.  Formes de emmagatzemar una matriu dispersa.  Implantar en Matlab les diferents formes de de fer-ho.
  %3.  Operacions  amb  matrius  disperses.   Suma,  Producte  matriu-vector,  Producte  matriu-matriu.
  %4.  Resoluci ́o de sistemes.  Reordenaci ́o de les matrius.
  %5.  Poseu exemples d’aplicaci ́o de la factoritzaci ́o LU aplicada a una sistema dispers.
  %6.  Doneu una estimaci ́o del estalvi en termes de temps de comput i espai

  \end{document}