\documentclass[11pt,a4paper,twoside]{report}
  \usepackage{a4wide}
  \usepackage{epsfig}
  \usepackage{amsmath}
  \usepackage{amsfonts}
  \usepackage{latexsym}
  \usepackage[utf8]{inputenc}
  \usepackage{listings}
  \usepackage{color}
  \usepackage{titlesec}    
  \usepackage{enumitem}
  \usepackage[catalan]{babel}
  \usepackage{newunicodechar}
  \usepackage{graphicx}
  \usepackage{subcaption}
  \usepackage{float}
  \newunicodechar{Ŀ}{\L.}
  \newunicodechar{ŀ}{\l.}
  
  
  % \titleformat{\chapter}
  %   {\normalfont\LARGE\bfseries}{\thechapter}{1em}{}
  % \titlespacing*{\chapter}{0pt}{3.5ex plus 1ex minus .2ex}{2.3ex plus .2ex}
  
  \definecolor{dkgreen}{rgb}{0,0.6,0}
  \definecolor{gray}{rgb}{0.5,0.5,0.5}
  \definecolor{mauve}{rgb}{0.58,0,0.82}
  
  \lstset{frame=tb,
    language=Matlab,
    aboveskip=3mm,
    belowskip=3mm,
    showstringspaces=false,
    columns=flexible,
    basicstyle={\small\ttfamily},
    numbers=none,
    numberstyle=\tiny\color{gray},
    keywordstyle=\color{blue},
    commentstyle=\color{dkgreen},
    stringstyle=\color{mauve},
    breaklines=true,
    breakatwhitespace=true,
    tabsize=3,
    extendedchars=true,
    literate={á}{{\'a}}1 {à}{{\`a}}1 {é}{{\'e}}1 {è}{{\`e}}1 {í}{{\'i}}1 {ó}{{\'o}}1 {ò}{{\`o}}1 {ú}{{\'u}}1
  }
  
  
  \usepackage{hyperref}
  \hypersetup{
      colorlinks=false, %set true if you want colored links
      linktoc=all,     %set to all if you want both sections and subsections linked
      linkcolor=blue,  %choose some color if you want links to stand out
  }
  
  
  \setlength{\footskip}{50pt}
  \setlength{\parindent}{0cm} \setlength{\oddsidemargin}{-0.5cm} \setlength{\evensidemargin}{-0.5cm}
  \setlength{\textwidth}{17cm} \setlength{\textheight}{23cm} \setlength{\topmargin}{-1.5cm} \addtolength{\parskip}{2ex}
  \setlength{\headsep}{1.5cm}
  
  \renewcommand{\contentsname}{Continguts}
  %\renewcommand{\chaptername}{Pr\`actica}
  \setcounter{chapter}{0}
  \begin{document}
  
  \title{Treball 1: Matrius disperses}
  \author{Ismael El Habri, Marc Cané, Lluís Trilla}
  \date{16 d'octubre de 2018}
  \maketitle
  
  \tableofcontents
  
  
  \chapter{Què són les matrius disperses?}
  
  Quan parlem de matrius disperses ens referim a matrius de gran tamany en la qual la majoria d'elements son zero. Direm que 
  una matriu és disperrsa, quan hi hagi benefici en aplicar els mètodes propis d'aquestes. 
  
  Per identificar si una matriu és dispersa, podem usar el seguent:
    
  \qquad Una matriu $n \times n$ serà dispersa si el número de coeficients no nuls es $n^{\gamma+1}$, on $\gamma < 1$.

  En funció del poblema, decidim el valor del paràmetre $\gamma$. Aquí hi ha els valors típics de $\gamma$:
  \begin{itemize}
    \item $\gamma=0.2$ per problemes d'anàlisi de sistemes eléctics degeneració i de transpot d'enegía.
    \item $\gamma=0.5$ per matrius en bandes associades a problemes d'anàlisi d'estructues.
  \end {itemize}



  Podem trobar dos tipus de matrius disperses:
  \begin{itemize}
    \item \textbf{matrius estructurades:} matrius en les quals els elements diferents de zero formen un patró regular. 
    \item \textbf{matrius no estructurades:} els elements diferents de zero es distribueixen de forma irregular.
  \end{itemize}


  
    
  \end{document}