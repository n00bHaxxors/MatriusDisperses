\documentclass{beamer}
\usetheme{Ilmenau}
\usecolortheme{dolphin}
  \usepackage{epsfig}
  \usepackage{amsmath}
  \usepackage{tabu}
  \usepackage{amsfonts}
  \usepackage{latexsym}
  \usepackage[utf8]{inputenc}
  \usepackage{listings}
  \usepackage[catalan]{babel}
  \usepackage{newunicodechar}
  \usepackage{graphicx}
  \usepackage{subcaption}
  \usepackage{float}
  \usepackage[numbered,framed]{matlab-prettifier}
  \usepackage{pgf, tikz}
  \usetikzlibrary{arrows, automata, positioning}

\newunicodechar{Ŀ}{\L.}
\newunicodechar{ŀ}{\l.}

\definecolor{dkgreen}{rgb}{0,0.6,0}
\definecolor{gray}{rgb}{0.5,0.5,0.5}
\definecolor{mauve}{rgb}{0.58,0,0.82}

\lstset{frame=tb,
language=Matlab,
aboveskip=3mm,
belowskip=3mm,
showstringspaces=false,
columns=flexible,
basicstyle={\small\ttfamily},
numbers=none,
numberstyle=\tiny\color{gray},
keywordstyle=\color{blue},
commentstyle=\color{dkgreen},
stringstyle=\color{mauve},
breaklines=true,
breakatwhitespace=true,
tabsize=3,
extendedchars=true,
literate={á}{{\'a}}1 {à}{{\`a}}1 {ã}{{\~a}}1 {é}{{\'e}}1 {è}{{\`e}}1 {í}{{\'i}}1 {ï}{{\"i}}1 {ó}{{\'o}}1 {ò}{{\`o}}1 {ú}{{\'u}}1 {ü}{{\"u}}1 {ç}{{\c{c}}}1
			{Á}{{\'A}}1 {À}{{\`A}}1 {Ã}{{\~A}}1 {É}{{\'E}}1 {È}{{\`E}}1 {Í}{{\'I}}1 {Ï}{{\"I}}1 {Ó}{{\'O}}1 {Ò}{{\`O}}1 {Ú}{{\'U}}1 {Ü}{{\"U}}1 {Ç}{{\c{C}}}1
}

\newcommand\double[3][10]{%Passantli A i B genera quatre vertexs virtuals A-B-s, A-B-e (per resepresentar una aresta) i B-A-s, B-A-e (per representar l'altre aresta)
  \draw (#2)
    edge [bend left=#1,draw=none]
    coordinate[at start](#2-#3-s)
    coordinate[at end](#2-#3-e)
    (#3)
    edge [bend right=#1,draw=none]
    coordinate[at start](#3-#2-e)
    coordinate[at end](#3-#2-s)
    (#3);
}

\title{Matrius Disperses}
\author{Marc Cané \and Ismael El Habri \and Lluís Trilla}

\date[KPT 2004] % (optional)
{7 de novembre de 2018}
\subject{Computació Numèrica i Simulació}

\AtBeginSection[]
{
  \begin{frame}
    \frametitle{Table of Contents}
    \tableofcontents[currentsection]
  \end{frame}
}
\begin{document}

\frame{\titlepage}

\section{Què són les matrius disperses?}
  \begin{frame}
    \frametitle{Què són les matrius disperses?}
    Quan parlem de matrius disperses ens referim a matrius de gran tamany en la qual la majoria d'elements son zero. Direm que 
una matriu és disperrsa, quan hi hagi benefici en aplicar els mètodes propis d'aquestes. 
  \end{frame}
  \begin{frame}
   Per identificar si una matriu és dispersa, podem usar el seguent:

Una matriu $n \times n$ serà dispersa si el número de coeficients no nuls es $n^{\gamma+1}$, on $\gamma < 1$.

En funció del poblema, decidim el valor del paràmetre $\gamma$. Aquí hi ha els valors típics de $\gamma$:]
\begin{itemize}
\item $\gamma=0.2$ per problemes d'anàlisi de sistemes eléctics degeneració i de transpot d'enegía.
\item $\gamma=0.5$ per matrius en bandes associades a problemes d'anàlisi d'estructues.
\end {itemize}
  \end{frame}
  
  \subsection{Tipus de matrius disperses}
  \begin{frame}
  \frametitle{Tipus de matrius disperses}
Podem trobar dos tipus de matrius disperses:
\begin{itemize}
\item \textbf{Matrius estructurades:} matrius en les quals els elements diferents de zero formen un patró regular. Exemple: Les matrius banda.
\item \textbf{Matrius no estructurades:} els elements diferents de zero es distribueixen de forma irregular.
\end{itemize}  
  \end{frame}
  
%
%Formes d'emmagatzemar
%  

\section[Emmagatzematge]{Formes d'emmagatzemar matrius disperses}
\subsection{Per Coordenades}
\begin{frame}
\frametitle{Per Coordenades}
És la primera aproximació que podríem pensar i és bastant intuïtiva. Per cada element no nul guardem una tupla amb el valor i les seves coordenades: $(a_{i j}, i, j)$. 

A la realitat però, aquest mètode d'emmagatzemar les dades és poc eficient quan hem de fer operacions amb les matrius.

\end{frame}
\begin{frame}
\frametitle{Exemple}
\[    
\begin{pmatrix}
	1	&	0	& 0	&	2	\\
	0	&	1	&	0	&	0	\\
	0	&	0	&	0	&	0	\\
	3	&	0	&	-2	&	0	\\
\end{pmatrix}   
\qquad = \qquad
    \begin{tabu}{l|c}
    	$índex$	&	$tupla$ (a_{i j}, i, j)	\\
    	\hline
    	0	&	(1, 1, 1) \\
    	1	&	(2, 1, 4) \\
    	2	& (1, 2, 2) \\
    	3	& (3, 4, 1) \\
    	4	& (-2, 4,3)	\\ 
    \end{tabu}  
\]
\end{frame}
\begin{frame}
Usem tres vectors de la mateixa mida ($n_z$, el nombre d'elements diferents de zero): Un amb els valors, un amb les files i un amb les columnes:
\begin{center}
	\begin{tabular}{l|c c c c c}
		Vector & \multicolumn{5}{c}{Coeficients}\\
		\hline
		valors			&	1	&	2	&	1 &	3	&	-2	\\
		files				&	1	&	1	&	2	&	4	&	4	\\
		columnes	&	1	&	4	&	2	&	1	&	3	\\ 	
	\end{tabular}	
\end{center}
\end{frame}


\subsection{Per Files}
\begin{frame}
\frametitle{Per Files}
També conegut com a \textit{Compressed Sparse Rows (CSR)}, \textit{Compressed Row Storage (CRS)}, o format \textit{Yale}. És el mètode més estès.

Consisteix en guardar els elements ordenats per files, guardar la columna on es troben, i la posició del primer element de cada fila en el vector de valors.
Així ens quedaran tres vectors:
\begin{itemize}
	\item \textbf{valors:} de mida $n_z$, conté tots els valors diferents.
	\item \textbf{columnes:} també de mida $n_z$, conté la columna on es troba cada un dels elements anteriors.
	\item \textbf{iniFiles:} de mida $m+1$, conté la posició on comença cada fila en els vectors valors i columnes, sent $m$ el nombre de files de la matriu. 
\end{itemize}

\end{frame}

\begin{frame}
\frametitle{Exemple}
\[    
\begin{pmatrix}
	1	&	0	& 0	&	2	\\
	0	&	1	&	0	&	0	\\
	0	&	0	&	0	&	0	\\
	3	&	0	&	-2	&	0	\\
\end{pmatrix}   \qquad = \qquad
\begin{tabu}{l|c c c c c}
		$Vector$ & \multicolumn{5}{c}{$Coeficients$}\\
		\hline
		$índex$			&	1	&	2	&	3	&	4	&	5	\\
		\hline
		$valors$			&	1	&	2	&	1 &	3	&	-2	\\
		$columnes$	&	1	&	4	&	2	&	1	&	3	\\ 	
		$iniFiles$			& 1	&	3	&	4	&	4 &	6 \\
\end{tabu}		\]

Si es canvien files per columnes, dona la implementació per columnes, o també anomenada \textit{Compressed Sparse Columns (CSC)}.

\end{frame}
  
\end{document}